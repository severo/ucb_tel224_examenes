\documentclass[addpoints]{exam}
\usepackage[utf8]{inputenc}

\boxedpoints
\pointpoints{punto}{puntos}
\vqword{Pregunta}
\vpgword{Página}
\vpword{Puntos}
\vsword{Nota}
\vtword{Total}


\begin{document}

\header{TEL224}{Primer control de lectura}{Viernes 06/03/2015}
\headrule


\title{Primer control de lectura - TEL224}
\date{Viernes 06/03/2015}
\maketitle

\begin{center}
\fbox{\fbox{\parbox{5.5in}{\centering
Responder a las preguntas en los espacios libres entre las preguntas. Si le falta espacio, escribir al reverso de la página.

Esta permitido el uso de apuntes y libros. Esta prohibido el uso de medios electrónicos, y el trabajo en equipo.}}}
\end{center}
\vspace{0.1in}
\makebox[\textwidth]{Nombre completo:\enspace\hrulefill}

\section*{Preguntas}

\begin{questions}
\question[20]
Determine si el sistema siguiente es (1) estable, (2) causal, (3) lineal y (4) invariante con el tiempo.

$$T(x[n]) = (\cos(\pi n))x[n]$$
\makeemptybox{\stretch{1}}
\newpage

\question[20]
La respuesta en frecuencia de un sistema lineal e invariante con el tiempo es

$$H\left(e^{j\omega}\right) = 1 - \frac{0.45 e^{-j\omega}}{1 - 0.8 e^{-j\omega}}$$

Especifique la ecuación en diferencias que relaciona la entrada $x[n]$ con la salida $y[n]$.

\makeemptybox{\stretch{1}}

\question[60]
Un sistema lineal, invariante con el tiempo y causal tiene como función de transferencia

$$H(z) = \frac{1-z^{-1}}{1- 0.25 z^{-2}} = \frac{1-z^{-1}}{\left(1- 0.5 z^{-1}\right)\left(1+0.5 z^{-1}\right)}$$

\begin{parts}
\part[20]

Defina la región de convergencia, y represente región de convergencia, ceros y polos en el plano z

\makeemptybox{\stretch{1}}
\newpage

\part[20]
Determine la salida del sistema cuando la entrada es $x[n] = u[n]$.

\makeemptybox{\stretch{1}}

\part[20]
Determine la entrada $x[n]$ de forma que la correspondiente salida del sistema anterior sea $y[n] = \delta[n] - \delta[n - 1].$


\makeemptybox{\stretch{1}}
\newpage
\end{parts}


\end{questions}

\section*{Resultados}

\begin{center}
\gradetable[v][questions]
\end{center}


\footrule
\footer{}{Página \thepage\ de \numpages}{}

\end{document}
