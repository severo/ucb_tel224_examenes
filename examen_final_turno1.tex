\documentclass[addpoints]{exam}
\usepackage[utf8]{inputenc}
\usepackage{amsmath}
\usepackage{mathrsfs}

\boxedpoints
\pointpoints{punto}{puntos}
\vqword{Pregunta}
\vpgword{Página}
\vpword{Puntos}
\vsword{Nota}
\vtword{Total}


\begin{document}

\header{TEL224}{Examen final primer turno / parte 1}{Viernes 19/06/2015}
\headrule


\title{Examen final primer turno / parte 1 - TEL224}
\date{Viernes 06/06/2015}
\maketitle

\vspace{0.1in}
\makebox[\textwidth]{Nombre completo:\enspace\hrulefill}

\section*{Pregunta 1}

Considere la sistema definido por la función de transferencia siguiente

$$H(z) = \frac{\left(z^{2} + 1\right)}{\left(z^{2}-0.3 z + 0.4\right)\left(1-0.8 e^{j\pi/3}z^{-1}\right)\left(1-0.8 e^{-j\pi/3}z^{-1}\right)}$$ con la región de convergencia $$|z| > 0.8$$

Calcular la ecuación en diferencias con coeficientes constantes del sistema.

Calcular los valores \(y[0]\), \(y[1]\), \(y[2]\) y \(y[3]\).

Encontrar los ceros y polos del sistema y dibujarlos en el plano z. 

¿El sistema es causal? 

¿El sistema es estable? 

¿El sistema es de fase mínima?

Dibujar el diagrama de bloques (grafo de flujo) del sistema en forma directa II.

Dibujar el diagrama de bloques (grafo de flujo) del sistema en forma cascada.

Dibujar el módulo de la respuesta en frecuencia del sistema. Es filtro pasa bajo? pasa banda? corta banda? pasa alto?

\footrule
\footer{}{Página \thepage\ de \numpages}{}

\end{document}
