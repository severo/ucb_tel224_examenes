\documentclass[addpoints]{exam}
\usepackage[utf8]{inputenc}
\usepackage{amsmath}
\usepackage{mathrsfs}

\boxedpoints
\pointpoints{punto}{puntos}
\vqword{Pregunta}
\vpgword{Página}
\vpword{Puntos}
\vsword{Nota}
\vtword{Total}


\begin{document}

\header{TEL224}{Segundo examen}{Viernes 08/05/2015}
\headrule


\title{Segundo examen - TEL224}
\date{Viernes 08/05/2015}
\maketitle

\vspace{0.1in}
\makebox[\textwidth]{Nombre completo:\enspace\hrulefill}

\section*{Preguntas}

\begin{questions}
\question[5]
¿Qué son las diferencias entre un conversor ideal de tiempo continuo a tiempo discreto y un conversor analógico-digital?
\makeemptybox{\stretch{1}}

\question[5]
Dibuje un sistema que permita multiplicar la frecuencia de muestreo por un factor \(1.2\) sin introducir solapamiento.
\makeemptybox{\stretch{1}}

\question[5]
Escribir el teorema de Nyquist
\makeemptybox{\stretch{1}}

\newpage

\question[5]
¿Que es la frecuencia de Nyquist?
\makeemptybox{\stretch{1}}

\question[5]
Considere el sistema que se muestra en la figura siguiente.

[PARA EL EXAMEN: Copiar la figura del ejercicio 4.15, pero con factor 2, en vez de 3.]

Para la señal de entrada \(x[n] = \text{sen}(\pi n / 3)\), calcule:

\begin{parts}

\part su transformada de Fourier \(X(e^{j\omega})\)

\makeemptybox{\stretch{1}}

\part la transformada de Fourier \(X_d(e^{j\omega})\)

\makeemptybox{\stretch{1}}

\newpage

\part la transformada de Fourier \(X_e(e^{j\omega})\)

\makeemptybox{\stretch{1}}

\part la transformada de Fourier \(X_r(e^{j\omega})\)

\makeemptybox{\stretch{1}}

\part la señal resultante \(x_r[n]\)

\makeemptybox{\stretch{1}}

\end{parts}

\newpage

\question[5]
Escriba la definición del retardo de grupo.

\makeemptybox{\stretch{1}}

\question[5]
¿Para que rango de frecuencias el módulo de la respuesta en frecuencia de un sistema pasa-todo es nulo?

\makeemptybox{\stretch{1}}

\question[5]
¿Por qué es generalmente deseable que un sistema de transmisión sea de fase lineal?

\makeemptybox{\stretch{1}}

\newpage

\question[5]
Para la función de transferencia \(H(z)\) de un sistema lineal e invariante en el tiempo, cite una condición suficiente sobre los polos, ceros y región de convergencia de \(H(z)\) para que:

\begin{parts}

\part el sistema sea estable

\makeemptybox{\stretch{1}}

\part el sistema sea causal

\makeemptybox{\stretch{1}}

\part el sistema sea de fase mínima

\makeemptybox{\stretch{1}}

\end{parts}
\newpage

\question[5]

Un sistema en tiempo discreto lineal, invariante con el tiempo y causal tiene como función de transferencia

\[
H(z) = \frac{\left(1 - 0.64 z^{-2}\right)\left(1 + 4 z^{-2}\right)}{\left(1 + 0.5 z^{-1}\right)}
\]

\begin{parts}
\part
Calcule y dibuje sus ceros, polos y región de convergencia.

\makeemptybox{\stretch{1}}

\part
Determine las expresiones de un sistema de fase mínima \(H_{min}(z)\) y de un sistema paso todo \(H_{ap}(z)\) de forma que

\[
H(z) = H_{min}(z) H_{ap}(z)
\]

\makeemptybox{\stretch{1}}

\newpage

\part
¿Es estable el sistema \(H_{ap}(z)\)?

\makeemptybox{\stretch{1}}
\part
¿Es causal el sistema \(H_{min}(z)\)?

\makeemptybox{\stretch{1}}

\end{parts}

\end{questions}

\section*{Resultados}

\begin{center}
\gradetable[v][questions]
\end{center}


\footrule
\footer{}{Página \thepage\ de \numpages}{}

\end{document}
