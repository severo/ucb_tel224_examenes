\documentclass[addpoints]{exam}
\usepackage[utf8]{inputenc}

\boxedpoints
\pointpoints{punto}{puntos}
\vqword{Pregunta}
\vpgword{Página}
\vpword{Puntos}
\vsword{Nota}
\vtword{Total}

\begin{document}

\header{TEL224}{Segundo control de lectura}{Viernes 24/04/2015}
\headrule

\title{Segundo control de lectura - TEL224}
\date{Viernes 24/04/2015}
\maketitle

\vspace{0.1in}
\makebox[\textwidth]{Nombre completo:\enspace\hrulefill}

\begin{questions}

\question

La señal en tiempo continuo

$$x_{c(t)} = \sin (200 \pi t) + sin (150 \pi t)$$

se muestra con periodo de muestreo \(T\) y se obtiene la señal en tiempo discreto

$$x[n] = \sin\left(\frac{\pi n}{3}\right) + \sin\left(\frac{\pi n}{4}\right)$$

\begin{parts}
\part[15]

Determine un valor de \(T\) que sea consistente con esta información.

\makeemptybox{\stretch{1}}

\part[15]

¿Es único el valor de \(T\) obtenido en la pregunta anterior? Si es así, explique por qué. Si no, indique otro valor de \(T\) que sea consistente con la información dada.

\makeemptybox{\stretch{1}}
\newpage
\end{parts}

\question

Cuando la entrada a un sistema lineal e invariante en el tiempo es

$$x[n] = \left(\frac{1}{6}\right)^{n} u[n] + \left(3\right)^{n} u[-n-1]$$

la salida es

$$y[n] = 5 \left(\frac{1}{6}\right)^{n} u[n] - 5 \left(\frac{1}{4}\right)^{n} u[n]$$

\begin{parts}
\part[20]

Determine la función de transferencia \(H(z)\) del sistema. Dibuje los polos y ceros de \(H(z)\), e indique la región de convergencia.

\makeemptybox{\stretch{1}}
\part[20]

Calcule la respuesta al impulso del sistema \(h[n]\) para todos los valores de \(n\).

\makeemptybox{\stretch{1}}
\newpage
\part[20]

Escriba la ecuación en diferencias que caracteriza el sistema.

\makeemptybox{\stretch{1}}
\part[10]

¿Es el sistema estable? ¿Es causal?

\makeemptybox{\stretch{1}}

\end{parts}


\end{questions}

\section*{Resultados}

\begin{center}
\gradetable[v][questions]
\end{center}


\footrule
\footer{}{Página \thepage\ de \numpages}{}

\end{document}
