\documentclass[addpoints]{exam}
\usepackage[utf8]{inputenc}
\usepackage{amsmath}
\usepackage{mathrsfs}

\boxedpoints
\pointpoints{punto}{puntos}
\vqword{Pregunta}
\vpgword{Página}
\vpword{Puntos}
\vsword{Nota}
\vtword{Total}


\begin{document}

\header{TEL224}{Tercer examen}{Viernes 06/06/2015}
\headrule


\title{Tercer examen - TEL224}
\date{Viernes 06/06/2015}
\maketitle

\vspace{0.1in}
\makebox[\textwidth]{Nombre completo:\enspace\hrulefill}

\section*{Pregunta 1}

Considere el sistema lineal, invariante con el tiempo, causal, con la función de transferencia siguiente

$$H(z) = \frac{\left(1 - 4 z^{-1} + 3 z^{-2}\right)\left(1+4z^{-2}\right)}{\left(4+z^{-2}\right)\left(9-z^{-2}\right)}$$

Encontrar los ceros y polos del sistema, la región de convergencia, y dibujarlos en el plano z. 

¿El sistema es estable? 

¿El sistema es de fase mínima?

Calcular la ecuación en diferencias con coeficientes constantes del sistema.

Descomponer la función de transferencia en \(H_{min}\) y \(H_{ap}\).

Dibujar el diagrama de bloques (grafo de flujo) de la forma cascada del sistema, con subbloques de orden 2.

Dibujar el diagrama de bloques (grafo de flujo) de la forma directa I del sistema.

Dibujar el diagrama de bloques (grafo de flujo) de la forma directa II del sistema.

Dibujar el diagrama de bloques (grafo de flujo) de la estructura traspuesta de la forma directa I del sistema.

\section*{Versión alternativa de la Pregunta 1}

Considere el sistema lineal, invariante con el tiempo, causal, con la función de transferencia siguiente

$$H(z) = \frac{\left(2 - 5 z^{-1} + 2 z^{-2}\right)\left(1+4z^{-2}\right)}{\left(4-z^{-2}\right)\left(9+z^{-2}\right)}$$

Mismas preguntas
\footrule
\footer{}{Página \thepage\ de \numpages}{}

\end{document}
