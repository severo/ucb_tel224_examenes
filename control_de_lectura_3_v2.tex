\documentclass[addpoints]{exam}
\usepackage[utf8]{inputenc}

\boxedpoints
\pointpoints{punto}{puntos}
\vqword{Pregunta}
\vpgword{Página}
\vpword{Puntos}
\vsword{Nota}
\vtword{Total}

\begin{document}

\header{TEL224}{Tercer control de lectura}{Viernes 29/05/2015}
\headrule

\title{Tercer control de lectura - TEL224}
\date{Viernes 29/05/2015}
\maketitle

\vspace{0.1in}
\makebox[\textwidth]{Nombre completo:\enspace\hrulefill}
\vspace{0.1in}

Considere el sistema lineal invariante con el tiempo con la función de transferencia siguiente

$$H(z) = \frac{1 + 2 z^{-1} + z^{-2}}{1 - \frac{1}{4} z^{-1} -\frac{3}{8} z^{-2}}$$

\begin{questions}

\question[20]

Dibuje el diagrama de bloques (grafo de flujo) de la forma directo I del sistema \(H(z)\).

\makeemptybox{\stretch{1}}
\question[20]


Dibuje el diagrama de bloques (grafo de flujo) de la forma directo II del sistema \(H(z)\).

\makeemptybox{\stretch{1}}
\newpage
\question[20]

Dibuje el diagrama de bloques (grafo de flujo) de la forma en cascada del sistema \(H(z)\).

\makeemptybox{\stretch{1}}
\question[20]

Dibuje el diagrama de bloques (grafo de flujo) de la forma en paralelo del sistema \(H(z)\).

\makeemptybox{\stretch{1}}
\newpage
\question[20]

Dibuje el diagrama de bloques (grafo de flujo) de la estructura traspuesta de la forma directa II del sistema \(H(z)\).

\makeemptybox{\stretch{1}}

\end{questions}

\section*{Resultados}

\begin{center}
\gradetable[v][questions]
\end{center}

\footrule
\footer{}{Página \thepage\ de \numpages}{}

\end{document}
